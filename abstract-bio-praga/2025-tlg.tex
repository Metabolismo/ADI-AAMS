\documentclass[11pt,a4paper]{article}
\usepackage[utf8]{inputenc}
\usepackage[T1]{fontenc}
\usepackage{alegreya}
\usepackage[margin=2.5cm]{geometry}
\usepackage{setspace}
\usepackage{titlesec}
\usepackage{microtype}

% Configurazione titoli
\titleformat{\section}{\large\bfseries}{\thesection}{1em}{}
\titleformat{\subsection}{\normalsize\bfseries}{\thesubsection}{1em}{}

% Rimozione numerazione pagine
\pagestyle{empty}

% Configurazione spaziatura
\onehalfspacing

\begin{document}

% Titolo
\begin{center}
{\large\textit{Through the Looking-Glass}}\\[0.5em]
{\Large\textbf{Interpretive Analysis: Bridging Epistemological Boundaries\\
in Contemporary Music Performance}}\\[1em]
\end{center}

% Abstract
\section*{Abstract}

This research addresses the epistemological separation between musical analysis and interpretive practice, proposing ``interpretive analysis'' as methodology emerging from direct engagement with vibrating matter. Applied to Domenico Guaccero's \textit{Luz} (from ``Descrizione del corpo''), this approach challenges traditional dichotomy between preliminary analysis and execution.

Grounded in phenomenology of interpretation, this research proposes unified methodological process where scientific frameworks (logic, analysis) and artistic practice (embodied perception, intuition) co-constitute reciprocally—precisely addressing the conference's exploration of parallel modes of inquiry between art and science.

Guaccero's \textit{Luz} articulates melodies through 24 timbral typologies, exposing timbral research to relationship with silence through ``audible animated silence.'' The methodology integrates spectromorphological tools, developing ``analytical listening grammar'' that emerges from technological mediation.

The approach employs ``instrumental hacking''—threshold procedures toward instrumental augmentation through feedback systems (internal microphones, external speakers, counter-airflow through bell-mounted speakers)—that emerged cybernetically from laboratory work. This developing methodology generates preliminary results including: timbral transformation mappings through Python/Mathematica audio descriptors and \LaTeX/TikZ synoptic tables, alternative notation systems departing from staff notation, and documentation protocols for interpretive events.

\textit{Luz} serves as primary case study due to its unique graphic precision, formal rigor, and timbral extension. The work's notation for ``grave instrument'' enables methodological transferability across different performers and instruments, potentially generating self-analyzing corpus of interpretations. The score's hybrid gesture/effect notation functions as interpretive hacking, encouraging discovery of unknown instrumental possibilities.

This research emerges from LEAP (Laboratorio ElettroAcustico Permanente, Rome), where collective laboratory practice provides methodological validation through group work and public engagement. Developed as doctoral research proposal, this work seeks to establish interpretive analysis as systematic methodology within academic frameworks. Remarkably, this methodology appears already encoded within \textit{Luz} itself, suggesting Guaccero's prescient understanding of interpretive research as collective knowledge-building.

The methodology offers pedagogical instruments overcoming theory/practice dichotomy in contemporary performer formation, fostering technical metron and conscious openness to interpretive act.

\vspace{1em}

% Bio section
\section*{Bio}

Alice Cortegiani completed her diploma in Clarinet and Chamber Music at the Conservatorio di Roma and specialized at the Accademia Nazionale di Santa Cecilia with A. Carbonare. Her concert activity has led her to perform in prestigious venues including Teatro Arsenale for La Biennale di Venezia, Cappella Paolina del Palazzo del Quirinale, Auditorium Parco della Musica di Roma, and Auditorio Nacional de Música de Madrid.

Her research focuses on musical artistic investigation through collaborations with composers and experimental laboratories, interrogating the interpreter's role as bridge between tradition and contemporaneity. This inquiry naturally converges with her involvement in LEAP (Laboratorio ElettroAcustico Permanente), where she develops methodologies integrating technological mediation with interpretive practice. Her pedagogical activity transmits both mastery of classical repertoire and openness toward interdisciplinary approaches, seeking to establish interpretive analysis as systematic methodology within academic frameworks.

\vspace{2em}

% Submission info
\begin{center}
\small
Submitted to: Art and Science: Thinking Outside the Box\\
Academy of Performing Arts in Prague (AMU)\\
November 18--20, 2025
\end{center}

\end{document}
