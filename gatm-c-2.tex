\documentclass[12pt,a4paper]{article}
\usepackage[utf8]{inputenc}
\usepackage[english,italian]{babel}
\usepackage{Alegreya}
\usepackage[T1]{fontenc}
\usepackage{geometry}
\usepackage{setspace}
\usepackage{titlesec}
\usepackage{enumitem}
\usepackage{natbib}

% Impostazioni pagina secondo le linee guida
\geometry{
    top=2.5cm,
    bottom=2.5cm,
    left=2.5cm,
    right=2.5cm
}

% Spaziatura
\onehalfspacing

% Formattazione titoli
\titleformat{\section}{\normalfont\large\bfseries}{\thesection}{1em}{}
\titleformat{\subsection}{\normalfont\normalsize\bfseries}{\thesubsection}{1em}{}

% Dati del convegno
\newcommand{\convegnotitolo}{XXII Convegno Internazionale di Analisi e Teoria Musicale}
\newcommand{\convegnodate}{2-5 ottobre 2025}
\newcommand{\convegnosede}{Conservatorio di Musica "G. Martucci", Salerno}

\begin{document}

% Header con informazioni convegno
\begin{center}
{\small \textbf{\convegnotitolo} \\
\convegnodate \\
\convegnosede}
\end{center}

\vspace{1cm}

% TITOLO DELLA PROPOSTA
\begin{center}
{\Large \textbf{Analisi interpretativa: \emph{Luz (da "Descrizione del corpo")}}}
\end{center}

\vspace{0.5cm}

% Informazioni autore (da rimuovere nella versione anonima)
\begin{center}
{\textit{[Nome Cognome]} \\
Affiliazione istituzionale \\
Email}
\end{center}

\vspace{1cm}

% ABSTRACT (massimo 500 parole)
\section*{Abstract}

% 1. Argomento e problema trattato
\textbf{Argomento e problema:} Il presente contributo è tratto dal percorso di ricerca che si articola dalla domanda \emph{come può l'interprete sviluppare una metodologia analitica che emerga dalla relazione diretta con la materia in vibrazione, piuttosto che precederla?} ed è riferito nello specifico all'\emph{analisi interpretativa} dell'opera \emph{Luz (da "Descrizione del corpo")} di Domenico Guaccero. L'atto interpretativo responsabile conduce verso le profondità della dimensione interpretativa stessa di \emph{Luz}, e sprigiona il nucleo del problema: la necessità di una prassi da ricostituire. L'\emph{analisi interpretativa} di \emph{Luz} si pone come necessità tra le possibili risposte a tale problema.

% 2. Background e stato dell'arte
\textbf{Background e stato dell'arte:} Il corpus di opere di Guaccero, da cui \emph{Luz} si dirama, si caratterizza per essere emersione di una pratica di ricerca artistica musicale filosoficamente informata ed intrisa di una acuta osservazione delle tessiture della propria contemporaneità, alimentata da un raffinato \emph{fare musicale} e azione sociale. In \emph{Luz} articola melodie di \emph{Timbri} di 24 differenti tipologie, con una scrittura che sintetizza oltre un decennio di ricerca grafica e disvela in sé una \emph{historìa} di letteratura utopica \citep{bachmann1993letteratura}. Espone la ricerca timbrica alla relazione con il silenzio, mediante un artificio compositivo geniale: l'introduzione di un silenzio \emph{udibile animato}. Nel metodo di studio introdotto analizzando il brano, timbro, silenzio e grafia musicale fondono il nucleo centrale della spirale speculativa di esperienza e conoscenza e alimentano l'individuazione di un nuovo grado di interpretazione, di un nuovo interprete. La ricerca musicale contemporanea ha prodotto contributi significativi nell'esplorazione timbrica, nell'organologia aumentata e nell'interpretazione come co-creazione \citep{guaccero1970improvvisazione}. La spettromorfologia \citep{smalley1997spectromorphology} ha fornito strumenti descrittivi sistematici. Tuttavia manca una sistematizzazione metodologica che integri queste prospettive nella prassi interpretativa tradizionale. L'atto interpretativo si scolpisce da cinque ascolti in concerto avvenuti nell'arco temporale di due anni. Il paradigma presentato vuole superare la tradizionale separazione tra analisi preliminare ed esecuzione, proponendo un processo conoscitivo che si compie attraverso l'atto interpretativo stesso \citep{hatten2021speculative}. La dimensione teorica si fonda su una fenomenologia dell'interpretazione \citep{rognoni1966fenomenologia} che integri il concetto di "misura" come \emph{metron} emergente di relazioni tra principio generatore e generati \citep{cacciari1996metafisica}.

% 3. Obiettivi dello studio
\textbf{Obiettivi:} La ricerca mira a: 1) ricongiungersi agli studi musicologici nella prospettiva di ricostituire nel concreto una prassi interpretativa per l'opera di Guaccero; 2) sistematizzare l'\emph{analisi interpretativa} come metodologia trasferibile; 3) sviluppare una \emph{grammatica dell'ascolto analitico} che documenti i processi cognitivi emergenti; 4) fornire strumenti pedagogici che superino la dicotomia teoria/prassi nella formazione dell'interprete contemporaneo.

% 4. Metodologia
\textbf{Metodologia:} L'\emph{analisi interpretativa} propone un processo metodologico unitario dove tre dimensioni si co-costituiscono reciprocamente: una \emph{grammatica dell'ascolto analitico} che emerge attraverso la mediazione tecnologica e che include mappature delle trasformazioni timbriche, sistemi di notazione delle relazioni emergenti tra gesto e suono, catalogazione dell'\emph{accadere} interpretativo che modifica la comprensione del materiale.

% 5. Risultati e Implicazioni
\textbf{Risultati e Implicazioni:} Un'analisi è riuscita quando produce nuove possibilità interpretative, quando apre il materiale musicale anziché chiuderlo in una interpretazione definitiva. La \emph{grammatica dell'ascolto analitico} non preesiste alla prassi ma si genera nell'\emph{hacking strumentale} con protocolli che costituiscono il dispositivo pedagogico che trasforma la tecnologia in estensione della corporeità interpretativa rendendo inoltre possibile un'attitudine di contributo-dialogo creativo nell'accesso alla fenomenologia dell'aumentazione. L'obiettivo finale è contribuire a un'\emph{archeologia del presente musicale} \citep{agamben2008apparatus}, dove l'interprete è mediatore tra tradizione e contemporaneità, sviluppando strumenti conoscitivi che trasformino i paradigmi didattici e colmino il divario tra ricerca extra-accademica e formazione istituzionale nella necessità di un pensiero che sappia abitare la tensione tra \emph{metron} tecnico e apertura consapevole dell'essere al dono dell'atto interpretativo.

\vspace{1cm}

% PAROLE CHIAVE
\textbf{Parole chiave:} analisi interpretativa, spettromorfologia, fenomenologia musicale, prassi esecutiva, Domenico Guaccero

\vspace{1cm}

\clearpage

% BIBLIOGRAFIA ESSENZIALE (massimo 5 titoli per abstract 500 parole)
\bibliographystyle{unsrt}
\bibliography{references}

\end{document}
