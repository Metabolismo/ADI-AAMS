\documentclass[12pt,a4paper]{article}
\usepackage[utf8]{inputenc}
\usepackage[english,italian]{babel}
\usepackage{Alegreya}
\usepackage[T1]{fontenc}
\usepackage{geometry}
\usepackage{setspace}
\usepackage{titlesec}
\usepackage{enumitem}
\usepackage{cite}
\usepackage{natbib}
\setstretch{1.2}
\usepackage{url}

% Impostazioni pagina secondo le linee guida
\geometry{
    top=2.5cm,
    bottom=2.5cm,
    left=2.5cm,
    right=2.5cm
}

% Spaziatura
\onehalfspacing

% Formattazione titoli
\titleformat{\section}{\normalfont\large\bfseries}{\thesection}{1em}{}
\titleformat{\subsection}{\normalfont\normalsize\bfseries}{\thesubsection}{1em}{}

% Dati del convegno
\newcommand{\convegnotitolo}{XXII Convegno Internazionale di Analisi e Teoria Musicale}
\newcommand{\convegnodate}{2-5 ottobre 2025}
\newcommand{\convegnosede}{Conservatorio di Musica "G. Martucci", Salerno}

\begin{document}

% Header con informazioni convegno
\begin{center}
{\small \textbf{\convegnotitolo} \\
\convegnodate \\
\convegnosede}
\end{center}

\vspace{1cm}

% TITOLO DELLA PROPOSTA
\begin{center}
{\Large \textbf{Analisi interpretativa: \emph{Luz (da "Descrizione del corpo")}}}
\end{center}

\vspace{0.5cm}

% Informazioni autore (da rimuovere nella versione anonima)
\begin{center}
{\textit{[Alice Cortegiani]} \\
Affiliazione istituzionale \\
Email}
\end{center}

\vspace{1cm}

% ABSTRACT (massimo 300 parole)
\section*{Abstract}

% 1. Argomento e problema trattato
Il presente contributo sviluppa un'\emph{analisi interpretativa} dell'opera \emph{Luz (da "Descrizione del corpo")} di Domenico Guaccero, emergendo dalla domanda fondamentale:\emph{come può l'interprete sviluppare una metodologia analitica che emerga dalla relazione diretta con la materia in vibrazione, piuttosto che precederla?} L'atto interpretativo responsabile conduce verso le profondità della dimensione interpretativa di \emph{Luz}, sprigionando il nucleo del problema: la necessità di ricostituire una prassi interpretativa.

% 2. Background e stato dell'arte
\emph{Luz} articola melodie di \emph{Timbri} attraverso 24 differenti tipologie, con una scrittura che sintetizza oltre un decennio di ricerca grafica e introduce un silenzio \emph{udibile animato}. La ricerca musicale contemporanea ha prodotto contributi significativi nell'esplorazione timbrica e nell'interpretazione come co-creazione \citep{hatten2021speculative}, mentre la spettromorfologia \citep{smalley1997spectromorphology} ha fornito strumenti descrittivi sistematici. Tuttavia manca una sistematizzazione metodologica che integri queste prospettive nella prassi interpretativa tradizionale.

% 3. Obiettivi dello studio
La ricerca mira a: 1) ricostituire concretamente una prassi interpretativa per l'opera di Guaccero; 2) sistematizzare l'\emph{analisi interpretativa} come metodologia trasferibile; 3) sviluppare una \emph{grammatica dell'ascolto analitico} che documenti i processi cognitivi emergenti.

% 4. Metodologia
L'\emph{analisi interpretativa} propone un processo unitario dove tre dimensioni si co-costituiscono: mappature delle trasformazioni timbriche, sistemi di notazione delle relazioni emergenti tra gesto e suono, catalogazione dell'\emph{accadere} interpretativo che modifica la comprensione del materiale.

% 5. Risultati
Un'analisi riesce quando produce nuove possibilità interpretative, aprendo il materiale musicale anziché chiuderlo. La \emph{grammatica dell'ascolto analitico} si genera attraverso protocolli che trasformano la tecnologia in estensione della corporeità interpretativa.

% 6. Implicazioni
L'obiettivo è contribuire a un'\emph{archeologia del presente musicale}, dove l'interprete media tra tradizione e contemporaneità, sviluppando strumenti conoscitivi che trasformino i paradigmi didattici nella tensione tra \emph{metron} tecnico \citep{cacciari1996metafisica} e apertura dell'essere all'atto interpretativo.

\vspace{1cm}

% PAROLE CHIAVE
\textbf{Parole chiave:} analisi interpretativa, spettromorfologia, fenomenologia musicale, prassi interpretativa, Domenico Guaccero

\vspace{1cm}



\bibliographystyle{unsrt}
\bibliography{references}


\end{document}
