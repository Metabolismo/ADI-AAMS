\documentclass[12pt,a4paper]{article}
\usepackage[utf8]{inputenc}
\usepackage[english,italian]{babel}
\usepackage{Alegreya}
\usepackage[T1]{fontenc}
\usepackage{geometry}
\usepackage{setspace}
\usepackage{titlesec}
\usepackage{enumitem}
\usepackage{cite}

% Impostazioni pagina secondo le linee guida
\geometry{
    top=2.5cm,
    bottom=2.5cm,
    left=2.5cm,
    right=2.5cm
}

% Spaziatura
\onehalfspacing

% Formattazione titoli
\titleformat{\section}{\normalfont\large\bfseries}{\thesection}{1em}{}
\titleformat{\subsection}{\normalfont\normalsize\bfseries}{\thesubsection}{1em}{}

% Dati del convegno
\newcommand{\convegnotitolo}{XXII Convegno Internazionale di Analisi e Teoria Musicale}
\newcommand{\convegnodate}{2-5 ottobre 2025}
\newcommand{\convegnosede}{Conservatorio di Musica "G. Martucci", Salerno}

\begin{document}

% Header con informazioni convegno
\begin{center}
{\small \textbf{\convegnotitolo} \\
\convegnodate \\
\convegnosede}
\end{center}

\vspace{1cm}

% TITOLO DELLA PROPOSTA
\begin{center}
{\Large \textbf{[INSERIRE TITOLO DELLA PROPOSTA]}}
\end{center}

\vspace{0.5cm}

% Informazioni autore (da rimuovere nella versione anonima)
\begin{center}
{\textit{[Nome Cognome]} \\
Affiliazione istituzionale \\
Email}
\end{center}

\vspace{1cm}

% ABSTRACT (massimo 300 parole)
\section*{Abstract}

% 1. Argomento e problema trattato
\textbf{Argomento e problema:} [Descrivere chiaramente l'argomento di ricerca e il problema specifico che si intende affrontare. Indicare se si tratta di analisi teorica, analitica di opere specifiche, aspetti strutturali o performativi.]

% 2. Background e stato dell'arte
\textbf{Background e stato dell'arte:} [Sintetizzare i principali contributi esistenti sull'argomento, citando 2-3 riferimenti essenziali che posizionano la ricerca nel contesto accademico attuale.]

% 3. Obiettivi dello studio
\textbf{Obiettivi:} [Specificare chiaramente cosa si intende raggiungere con questo studio, quali domande di ricerca si vogliono rispondere.]

% 4. Metodologia
\textbf{Metodologia:} [Descrivere il metodo di analisi musicale utilizzato, gli strumenti teorici impiegati, l'approccio metodologico adottato.]

% 5. Risultati
\textbf{Risultati:} [Presentare i principali risultati ottenuti o quelli che si prevede di raggiungere, evidenziando gli aspetti più significativi.]

% 6. Implicazioni
\textbf{Implicazioni:} [Spiegare quale contributo questo studio apporta alla letteratura esistente, quali sono le possibili applicazioni e l'impatto nella comunità scientifica.]

\vspace{1cm}

% PAROLE CHIAVE
\textbf{Parole chiave:} [parola1], [parola2], [parola3], [parola4], [parola5]

\vspace{1cm}

\clearpage

% BIBLIOGRAFIA ESSENZIALE (massimo 5 titoli)
\section*{Bibliografia}

\begin{enumerate}[label={[\arabic*]}]
    \item [Autore, Anno. \textit{Titolo}. Editore.]
    \item [Autore, Anno. "Titolo articolo". \textit{Rivista}, vol. X, pp. xx-xx.]
    \item [Autore, Anno. \textit{Titolo opera}. Editore.]
    \item [Autore, Anno. "Titolo contributo". In \textit{Titolo raccolta}, a cura di Editor, pp. xx-xx. Editore.]
    \item [Autore, Anno. \textit{Titolo}. Editore.]
\end{enumerate}

\newpage

% SUGGERIMENTI E PROMEMORIA
\section*{Promemoria per la compilazione}

\subsection*{Date importanti:}
\begin{itemize}
    \item \textbf{10 giugno 2025:} Scadenza invio proposte
    \item \textbf{10 luglio 2025:} Comunicazione risultati
    \item \textbf{11-31 luglio 2025:} Registrazione anticipata (quota ridotta)
    \item \textbf{1-25 agosto 2025:} Registrazione quota intera
\end{itemize}

\subsection*{Specifiche tecniche:}
\begin{itemize}
    \item \textbf{Abstract:} massimo 300 parole
    \item \textbf{Proposta completa:} massimo 500 parole + 2 pagine esempi/figure
    \item \textbf{Bibliografia:} massimo 5 titoli
    \item \textbf{Parole chiave:} esattamente 5
    \item \textbf{Lingue ufficiali:} italiano e inglese
\end{itemize}

\subsection*{Tipologie di presentazione:}
\begin{enumerate}
    \item \textbf{Relazione individuale:} 20 min + 10 min discussione
    \item \textbf{Sessione preorganizzata:} 3-4 relazioni da 20 min ciascuna
    \item \textbf{Relazione-concerto:} 30 min totali + 10 min discussione
\end{enumerate}

\subsection*{Criteri di valutazione (scala 1-5):}
\begin{itemize}
    \item Coerenza con il tema scelto
    \item Consistenza e pertinenza del background teorico
    \item Chiarezza degli obiettivi
    \item Rigore metodologico
    \item Originalità e interesse dei risultati
    \item Potenziale impatto nella comunità scientifica
    \item Chiarezza del linguaggio e stile di scrittura
\end{itemize}

\subsection*{Suggerimenti per aree di ricerca innovative:}
\begin{itemize}
    \item \textbf{Tecnologie digitali:} Analisi assistita da computer, sintesi audio, live electronics
    \item \textbf{Approcci interdisciplinari:} Filosofia della musica, estetica, fenomenologia
    \item \textbf{Performance studies:} Rapporto analisi-esecuzione, interpretazione storicamente informata
    \item \textbf{Musicologie computazionali:} Machine learning, corpus studies, digital humanities
    \item \textbf{Teoria post-tonale:} Analisi spettrali, microtonalità, computer music
\end{itemize}

\subsection*{Invio proposta:}
\begin{itemize}
    \item \textbf{Modulo Google:} \texttt{https://forms.gle/rxqUnbBTp4QnxmPP6}
    \item \textbf{Contatti:} \texttt{convegnoannuale@gatm.it}
    \item \textbf{Sito web:} \texttt{https://www.gatm.it}
\end{itemize}

\end{document}
