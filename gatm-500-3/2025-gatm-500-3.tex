\documentclass[11pt]{article}
\usepackage[utf8]{inputenc}
\usepackage[T1]{fontenc}
\usepackage[italian]{babel}
\usepackage{Alegreya}
\usepackage[margin=2.5cm]{geometry}
\usepackage{setspace}
\usepackage{natbib}
\setstretch{1.2}
\usepackage{url}
\usepackage{graphicx}

\title{\textbf{Analisi interpretativa: \emph{Luz (da "Descrizione del corpo")}}}


\begin{document}

\maketitle
%ARGOMENTO E PROBLEMA
%\textbf{Argomento e problema:}
Il presente contributo affronta la separazione epistemologica tra analisi musicale e prassi interpretativa. Partendo dalla domanda \emph{come può l'interprete sviluppare una metodologia analitica che emerga dalla relazione diretta con la materia in vibrazione, anziché precederla?}, la ricerca propone il concetto di \textit{analisi interpretativa} applicata all'opera \emph{Luz (da "Descrizione del corpo")} di Domenico Guaccero. Il problema centrale riguarda la necessità di sistematizzare una coscienza interpretativa che sia simultaneamente prassi riflessiva e azione trasformativa, di fondamentale importanza per quella letteratura la cui prassi interpretativa è da ri-attivare.


%BACKGROUND E STATO DELL'ARTE
%\textbf{Background e stato dell'arte:}
Il corpus di opere di Guaccero, da cui \emph{Luz} si dirama, emerge da una pratica di ricerca artistica musicale filosoficamente informata e osserva acutamente i molteplici aspetti della propria contemporaneità, alimentata da una raffinato \emph{fare musicale} e azione sociale. In \emph{Luz} articola melodie di \emph{Timbri} di 24 tipologie, una scrittura che sintetizza oltre un decennio di ricerca grafica e disvela in sé una \emph{historìa} di \emph{letteratura utopica} \citep{bachmann1993letteratura}. Espone la ricerca timbrica alla relazione con il silenzio, mediante un artificio compositivo: l'introduzione di un silenzio \emph{udibile animato}. Nel metodo di studio introdotto analizzando il brano, \emph{timbro}, \emph{silenzio} e \emph{grafia musicale} fondono il nucleo centrale della spirale speculativa di esperienza e conoscenza, alimentando l'individuazione di un nuovo interprete. La ricerca musicale contemporanea ha prodotto contributi significativi nell'esplorazione timbrica e nell'interpretazione come co-creazione. La spettromorfologia \citep{smalley1997spectromorphology} fornisce strumenti descrittivi sistematici. Tuttavia, manca una sistematizzazione metodologica che integri queste prospettive nella prassi interpretativa tradizionale.
L'atto interpretativo in forma di analisi di \emph{Luz} si scolpisce dalla sala da concerto. %Considera le astrazioni delle scienze particolari per considerare, insieme ad esse, anche quella dimensione della realtà e dell'essente che riguarda il rapporto con la parte ed il tutto. Sorge dagli interrogativi della realtà quotidiana dell'interprete-in-quanto-essente in relazione alla concretezza assoluta dell'invisibile, raccogliendo e collegando l'\emph{unità di molti}, l'essenza complessa di elementi di cui è costituito il \emph{molteplice}, in continuo divenire.\citep{cacciari1996metafisica}
Il paradigma presentato vuole superare la separazione tra analisi preliminare ed esecuzione, proponendo un processo conoscitivo che si compie attraverso l'atto interpretativo stesso \citep{hatten2021speculative}. La dimensione teorica si fonda su una fenomenologia dell'interpretazione \citep{rognoni1966fenomenologia} che integri il concetto di "misura" come \textit{metron} emergente di relazioni tra principio generatore e generati \citep{cacciari1996metafisica}, l'approccio alla prassi dei laboratori sperimentali e la metodologia compositiva.

%OBIETTIVI
%\textbf{Obiettivi:}
La ricerca mira a: 1) ricongiungersi agli studi musicologici per ricostituire una prassi interpretativa per l'opera di Guaccero 2) sistematizzare l'\textit{analisi interpretativa} come metodologia trasferibile; 3) fornire strumenti pedagogici che superino la dicotomia teoria/prassi nella formazione dell'interprete contemporaneo.
%METODOLOGIA
%\textbf{Metodologia:}
L'\textit{analisi interpretativa} propone un processo metodologico unitario dove più dimensioni si co-costituiscono reciprocamente: una \emph{grammatica dell'ascolto analitico} emergente dalla mediazione tecnologica e che include mappature delle trasformazioni timbriche, sistemi di notazione delle relazioni emergenti tra gesto e suono, catalogazione dell'\emph{accadere} interpretativo che modifica la comprensione del materiale.

%RISULTATI - IMPLICAZIONI
%\textbf{Risultati e Implicazioni:}
Un'analisi riesce quando produce nuove possibilità interpretative, quando apre il materiale musicale anziché chiuderlo in una interpretazione definitiva.
La \emph{grammatica dell'ascolto analitico} non preesiste alla prassi ma si genera nell'\emph{hacking strumentale} con protocolli che costituiscono il dispositivo pedagogico che trasforma la tecnologia in estensione della corporeità interpretativa, rendendo inoltre possibile un'attitudine di contributo-dialogo creativo nell'accesso alla fenomenologia dell'aumentazione.
L'obiettivo finale è contribuire a un'\emph{archeologia del presente musicale}, dove l'interprete è mediatore tra tradizione e contemporaneità, sviluppando strumenti conoscitivi che trasformino i paradigmi didattici, colmando il divario tra ricerca extra-accademica e formazione istituzionale nella necessità di un pensiero che sappia abitare la tensione tra \textit{metron} tecnico e apertura consapevole dell'essere al dono dell'atto interpretativo.

\vspace{1cm}

% PAROLE CHIAVE
\textbf{Parole chiave:} analisi interpretativa, spettromorfologia, fenomenologia musicale, prassi interpretativa, Domenico Guaccero

\vspace{1cm}

%\begin{figure}[htbp]
%    \centering
%    \includegraphics[width=0.7\textwidth]{nome_immagine.pdf}
%    \caption{Didascalia dell'immagine}
%    \label{fig:etichetta}
%\end{figure}


\bibliographystyle{unsrt}
\bibliography{references}

\end{document}
