\documentclass[11pt]{article}
\usepackage[utf8]{inputenc}
\usepackage[T1]{fontenc}
\usepackage[italian]{babel}
\usepackage{Alegreya}
\usepackage[margin=2.5cm]{geometry}
\usepackage{setspace}
\usepackage{natbib}
\setstretch{1.2}
\usepackage{url}

\title{\textbf{Autocostruzione di un Interprete:\\Analisi in ascolto della materia sonora\\ Luz (da "Descrizione del corpo")}}
%\author{Alice Cortegiani}
\date{\today}

\begin{document}

\maketitle

La presente ricerca tesse il concetto di \textit{analisi interpretativa} scolpito dalla domanda \emph{come può l'interprete sviluppare una metodologia analitica che emerga dalla relazione diretta con la materia sonora, piuttosto che precederla?}\\
Il paradigma presentato vuole superare la tradizionale separazione tra analisi preliminare ed esecuzione, proponendo un processo conoscitivo che si compie attraverso l'atto interpretativo stesso \citep{hatten2021speculative}. La dimensione teorica si fonda su una fenomenologia dell'interpretazione \citep{rognoni1966fenomenologia, ihde2007listening, merleau-ponty1945phenomenologie} che integri il concetto di "misura" come \textit{metron} emergente di relazioni tra principio generatore e generati \citep{cacciari1996metafisica}, l'approccio alla materia sonora dei laboratori sperimentali e la metodologia esplorativa compositiva.

Laddove un'analisi tradizionale \emph{dell'oggetto} mantiene una distanza epistemologica e "mette tra parentesi" l'esperienza vissuta per oggettivare il materiale musicale, l'\textit{analisi interpretativa} si configura come \emph{praxis nell'ascolto}, % dove il momento conoscitivo coincide con l'atto interpretativo stesso.
e mantiene la tensione tra \emph{noesis} e \emph{noema} \citep{husserl1913ideen}, facendo dell'interpretazione stessa il luogo della conoscenza. L'obiettivo è la sistematizzazione di una coscienza interpretativa che sia simultaneamente prassi riflessiva e azione trasformativa, superando la dicotomia soggetto-oggetto attraverso la comunione di \textit{Physis} e \textit{Logos} \citep{cacciari1991labirinto}.

L'atto interpretativo in forma di ricerca è in questo caso riferito a Luz \citep{guaccero1973luz} di Domenico Guaccero, per la cui imponenza teorica
si dispone a manuale operativo. \emph{Luz} articola melodie di \emph{Timbri} di 24 differenti tipologie, con una scrittura che sintetizza oltre un decennio di ricerca grafica e disvela in sé una \emph{historìa} di \emph{letteratura utopica} \citep{bachmann1993letteratura}. Espone la ricerca timbrica alla relazione con il silenzio, mediante un artificio compositivo geniale: l'introduzione di un silenzio \emph{udibile animato}. Nel metodo di studio introdotto analizzando il brano, timbro, silenzio e grafia musicale fondono il nucleo centrale della spirale speculativa di esperienza e conoscenza e alimentano l'individuazione di un nuovo grado di interpretazione, di un nuovo interprete.


Il contributo di questa ricerca consiste nella sistematizzazione dell'\textit{analisi interpretativa} come metodologia trasferibile per la formazione dell'interprete contemporaneo %Partendo dall'analisi di \textit{Luz} \citep{guaccero1973luz}, dalla collaborazione alla realizzazione di strumenti aumentati e dall'esperienza laboratoriale, il progetto propone
proponendo un processo metodologico unitario dove tre dimensioni si co-costituiscono reciprocamente: una \emph{grammatica dell'ascolto analitico} che emerge attraverso la mediazione tecnologica %(documenta i processi cognitivi dell'\textit{analisi interpretativa}, superando l'empirismo della tradizione orale e integrando spettromorfologia, fenomenologia e analisi timbrica.
e che include mappature delle trasformazioni timbriche, sistemi di notazione delle relazioni emergenti tra gesto e suono, catalogazione dell'\emph{accadere} interpretativo che modifica la comprensione del materiale. Questa grammatica non preesiste alla prassi ma si genera nell'\emph{hacking strumentale} con protocolli che costituiscono il dispositivo pedagogico che trasforma la tecnologia in estensione della corporeità interpretativa rendendo inoltre possibile l'accesso alla fenomenologia dell'aumentazione. % - trasformazione della corporeità vissuta che rende trasmissibile l'esperienza.
Un'analisi è riuscita quando produce nuove possibilità interpretative, quando apre il materiale musicale anziché chiuderlo in una interpretazione definitiva. %La \emph{pedagogia dell'autocostruzione} si fonda sulla documentazione dei processi attivati dalla pratica aumentata \citep{freire1970pedagogia} %: mediante laboratori di \emph{ascolto attivo}; %dove l'analisi emerge attraverso un rinnovato accesso alla letteratura
%laboratori di \emph{hacking} strumentale e approcci all'aumentazione; la stesura di un metodo di documentazione riflessiva. Questa pedagogia
%e \emph{profana} i dispositivi didattici tradizionali, rendendo inoperosa la separazione tra teoria e prassi per aprire nuove possibilità formative.

L'obiettivo finale è contribuire a un'\emph{archeologia del presente musicale} \citep{agamben2008apparatus}, dove l'interprete è mediatore tra tradizione e contemporaneità, sviluppando strumenti conoscitivi che trasformino i paradigmi didattici e colmino il divario tra ricerca extra-accademica e formazione istituzionale nella necessità di un pensiero che sappia abitare la tensione tra \textit{metron} tecnico e apertura dell'essere nella materia.



%\emph{Esiste una dimensione teorica che integri fenomenologia dell'interpretazione, teoria del timbro e metodologie di hacking strumentale per sistematizzare una coscienza interpretativa trasformativa?}

%La tradizione interpretativa occidentale si fonda su una separazione epistemologica tra momento analitico e momento esecutivo, %eredità logocentrica che privilegia la comprensione intellettuale rispetto all'esperienza fenomenologica del suono.
%La ricerca musicale contemporanea ha prodotto contributi significativi nell'esplorazione timbrica \citep{mcadams2022perception}, nell'organologia aumentata \citep{magnusson2009hermeneutic, lupone2003feedback} e nell'interpretazione come co-creazione \citep{guaccero1970improvvisazione}. La spettromorfologia \citep{smalley1997spectromorphology} ha fornito strumenti descrittivi sistematici, tuttavia manca una sistematizzazione metodologica che integri queste prospettive nella prassi interpretativa tradizionale.

%\emph{Come si re-integra la ricerca sull'\textit{analisi interpretativa} nei sistemi tradizionali di formazione dell'interprete?}



\bibliographystyle{unsrt}
\bibliography{references}

\end{document}
