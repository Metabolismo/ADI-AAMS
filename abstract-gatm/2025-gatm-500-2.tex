\documentclass[11pt]{article}
\usepackage[utf8]{inputenc}
\usepackage[T1]{fontenc}
\usepackage[italian]{babel}
\usepackage{Alegreya}
\usepackage[margin=2.5cm]{geometry}
\usepackage{setspace}
\usepackage{natbib}
\setstretch{1.2}
\usepackage{url}

\title{\textbf{Analisi interpretativa: Luz (da "Descrizione del corpo")}}
%\author{Alice Cortegiani}
\date{\today}

\begin{document}

\maketitle
%ARGOMENTO E PROBLEMA
Il presente contributo è tratto dal percorso di ricerca che si articola dalla domanda \emph{come può l'interprete sviluppare una metodologia analitica che emerga dalla relazione diretta con la materia sonora, piuttosto che precederla?}\\ ed è riferito nello specifico all'\textit{analisi interpretativa} dell'opera \emph{Luz (da "Descrizione del corpo")} di Domenico Guaccero.
Nell'avvicendarsi verso le profondità della dimensione interpretativa di \emph{Luz}, emerge la problematica di una prassi da ricostituire. L'\textit{analisi interpretativa} proposta si pone come necessità tra le possibili risposte a tale problema.\\  %da riarticolare nello stile quest'ultima frase... considerare ad esempio di parlarne inserendo Luz come caso di quelle opere che necessitano di... perché... che condividono il problema. Insomma parlarne da sguardo più ampio nella consapevolezza, anche se da "risolvere" con una frase semplice.

%BACKGROUND E STATO DELL'ARTE
Il corpus di opere di Guaccero, da cui \emph{Luz} si dirama, si caratterizza per essere emersione di una pratica di ricerca artistica (e) musicale filosoficamente informata ed intrisa di una vivida e acuta osservazione delle plurime tessiture della propria contemporaneità, alimentata da una raffinato \emph{fare musicale} e azione sociale. In \emph{Luz} articola melodie di \emph{Timbri} di 24 differenti tipologie, con una scrittura che sintetizza oltre un decennio di ricerca grafica e disvela in sé una \emph{historìa} di \emph{letteratura utopica} \citep{bachmann1993letteratura}. Espone la ricerca timbrica alla relazione con il silenzio, mediante un artificio compositivo geniale: l'introduzione di un silenzio \emph{udibile animato}. Nel metodo di studio introdotto analizzando il brano, timbro, silenzio e grafia musicale fondono il nucleo centrale della spirale speculativa di esperienza e conoscenza e alimentano l'individuazione di un nuovo grado di interpretazione, di un nuovo interprete.La ricerca musicale contemporanea ha prodotto contributi significativi nell'esplorazione timbrica, nell'organologia aumentata \citep{magnusson2009hermeneutic, lupone2003feedback} e nell'interpretazione come co-creazione \citep{guaccero1970improvvisazione}. La spettromorfologia \citep{smalley1997spectromorphology} ha fornito strumenti descrittivi sistematici, tuttavia manca una sistematizzazione metodologica che integri queste prospettive nella prassi interpretativa tradizionale.\\

%OBIETTIVI
L'atto interpretativo in forma di analisi di \emph{Luz} si scolpisce nella \emph{durata} di cinque ascolti in concerto calendarizzati nell'arco di due anni. Considera le astrazioni delle scienze particolari per considerare, insieme ad esse, anche quella dimensione della realtà e dell'essente che riguardda il rapporto con la parte ed il tutto. Sorge dagli interrogativi della realtà quotidiana dell'interprete in quanto essente in relazione alla concretezza assoluta dell'invisibile, raccogliendo e collegando l'\emph{unità di molti}, l'essenza complessa di elementi di cui è costituito il \emph{molteplice}, in continuo divenire.
Il paradigma presentato vuole superare la tradizionale separazione tra analisi preliminare ed esecuzione, proponendo un processo conoscitivo che si compie attraverso l'atto interpretativo stesso \citep{hatten2021speculative}. La dimensione teorica si fonda su una fenomenologia dell'interpretazione \citep{rognoni1966fenomenologia, ihde2007listening, merleau-ponty1945phenomenologie} che integri il concetto di "misura" come \textit{metron} emergente di relazioni tra principio generatore e generati \citep{cacciari1996metafisica}, l'approccio alla materia sonora dei laboratori sperimentali e la metodologia esplorativa compositiva.

%METODOLOGIA
L'\textit{analisi interpretativa} propone un processo metodologico unitario dove tre dimensioni si co-costituiscono reciprocamente: una \emph{grammatica dell'ascolto analitico} che emerge attraverso la mediazione tecnologica e che include mappature delle trasformazioni timbriche, sistemi di notazione delle relazioni emergenti tra gesto e suono, catalogazione dell'\emph{accadere} interpretativo che modifica la comprensione del materiale.

%RISULTATI - IMPLICAZIONI
Questa grammatica non preesiste alla prassi ma si genera nell'\emph{hacking strumentale} con protocolli che costituiscono il dispositivo pedagogico che trasforma la tecnologia in estensione della corporeità interpretativa rendendo inoltre possibile un'attitudine di contributo-dialogo creativo nell'accesso alla fenomenologia dell'aumentazione.
Un'analisi è riuscita quando produce nuove possibilità interpretative, quando apre il materiale musicale anziché chiuderlo in una interpretazione definitiva.
L'obiettivo finale è contribuire a un'\emph{archeologia del presente musicale} \citep{agamben2008apparatus}, dove l'interprete è mediatore tra tradizione e contemporaneità, sviluppando strumenti conoscitivi che trasformino i paradigmi didattici e colmino il divario tra ricerca extra-accademica e formazione istituzionale nella necessità di un pensiero che sappia abitare la tensione tra \textit{metron} tecnico e apertura consapevole dell'essere al dono dell'atto interpretativo.



\bibliographystyle{unsrt}
\bibliography{references1}

\end{document}
