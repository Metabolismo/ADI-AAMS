\documentclass[11pt,a4paper]{article}
\usepackage[utf8]{inputenc}
\usepackage[T1]{fontenc}
\usepackage[italian]{babel}
\usepackage{Alegreya}
\usepackage[margin=2.5cm]{geometry}
\usepackage{setspace}

% Impostazioni tipografiche
\onehalfspacing
\setlength{\parindent}{0pt} % Nessuna indentazione
\setlength{\parskip}{1em} % Spazio tra paragrafi

% Rimuovi numerazione pagine
\pagestyle{empty}

% Comando per il nome (più grande e in grassetto)
\newcommand{\bioname}[1]{%
    \begin{center}
        {\fontsize{18pt}{21.6pt}\selectfont\textbf{#1}}
    \end{center}
    \vspace{0.3em}
}

% Comando per l'affiliazione
\newcommand{\bioaffiliation}[1]{%
    \begin{center}
        {\fontsize{12pt}{14.4pt}\selectfont\textit{#1}}
    \end{center}
    \vspace{1.5em}
}

\begin{document}

% Nome autore
\bioname{Alice Cortegiani}

% Affiliazione
\bioaffiliation{LEAP Laboratorio ElettroAcustico Permanente}

% Corpo della biografia (inserire qui il testo di 150 parole)
Si diploma in Clarinetto e Musica da Camera presso il Conservatorio di Roma e si perfeziona presso l’Accademia Nazionale di Santa Cecilia con A. Carbonare. La sua attività concertistica l’ha condotta a esibirsi in prestigiose sedi quali Teatro Arsenale per La Biennale di Venezia, Cappella Paolina del Palazzo del Quirinale, Auditorium Parco della Musica di Roma, Teatro dell’Opera di Roma e Auditorio Nacional de Musica de Madrid. La sua attitudine è rivolta verso la ricerca artistica musicale attraverso collaborazioni con compositori e laboratori di sperimentazione, interrogando il ruolo dell’interprete come ponte tra tradizione e con- temporaneità. Questa ricerca confluisce naturalmente nella sua attività didattica, dove trasmette tanto la padronanza del repertorio classico quanto l’apertura verso altri linguaggi.

\end{document}
