\documentclass[12pt,a4paper]{article}
\usepackage[utf8]{inputenc}
\usepackage[english,italian]{babel}
\usepackage{Alegreya}
\usepackage[T1]{fontenc}
\usepackage{geometry}
\usepackage{setspace}
\usepackage{titlesec}
\usepackage{enumitem}
\usepackage{cite}

% Impostazioni pagina secondo le linee guida
\geometry{
    top=2.5cm,
    bottom=2.5cm,
    left=2.5cm,
    right=2.5cm
}

% Spaziatura
\onehalfspacing

% Formattazione titoli
\titleformat{\section}{\normalfont\large\bfseries}{\thesection}{1em}{}
\titleformat{\subsection}{\normalfont\normalsize\bfseries}{\thesubsection}{1em}{}

% Dati del convegno
\newcommand{\convegnotitolo}{XXII Convegno Internazionale di Analisi e Teoria Musicale}
\newcommand{\convegnodate}{2-5 ottobre 2025}
\newcommand{\convegnosede}{Conservatorio di Musica "G. Martucci", Salerno}

\begin{document}

% Header con informazioni convegno
\begin{center}
{\small \textbf{\convegnotitolo} \\
\convegnodate \\
\convegnosede}
\end{center}

\vspace{1cm}

% TITOLO DELLA PROPOSTA
\begin{center}
{\Large \textbf{Autocostruzione di un Interprete: Analisi Interpretativa della Materia Sonora in \textit{Luz} di Domenico Guaccero}}
\end{center}

\vspace{0.5cm}

% Informazioni autore (da rimuovere nella versione anonima)
\begin{center}
{\textit{[Nome Cognome]} \\
Affiliazione istituzionale \\
Email}
\end{center}

\vspace{1cm}

% ABSTRACT VERSIONE 500 PAROLE
\section*{Abstract}

% 1. Argomento e problema trattato
\textbf{Argomento e problema:} La ricerca affronta il problema della separazione epistemologica tra analisi musicale ed esecuzione interpretativa nella prassi contemporanea. Propone il concetto di \textit{analisi interpretativa} come metodologia unificata dove la conoscenza del materiale musicale emerge attraverso l'atto interpretativo stesso, superando la tradizionale dicotomia teoria/prassi. L'obiettivo è sistematizzare una coscienza interpretativa che sia simultaneamente riflessione analitica e azione trasformativa, sviluppando strumenti conoscitivi per la formazione dell'interprete contemporaneo attraverso l'analisi fenomenologica di \textit{Luz} di Domenico Guaccero.

% 2. Background e stato dell'arte
\textbf{Background e stato dell'arte:} La spettromorfologia \cite{smalley1997spectromorphology} ha sistematizzato l'analisi del suono elettroacustico, mentre la fenomenologia dell'interpretazione \cite{rognoni1966fenomenologia, ihde2007listening} ha aperto prospettive sull'esperienza vissuta dell'ascolto. La ricerca sull'organologia aumentata \cite{lupone2003feedback} ha trasformato la relazione interprete-strumento. Tuttavia, manca una metodologia che integri questi approcci nella prassi interpretativa tradizionale, colmando il divario tra ricerca extra-accademica e formazione istituzionale.

% 3. Obiettivi dello studio
\textbf{Obiettivi:} Lo studio mira a: 1) sistematizzare l'\textit{analisi interpretativa} come metodologia trasferibile; 2) sviluppare una ``grammatica dell'ascolto analitico'' che documenti i processi cognitivi emergenti; 3) creare protocolli di hacking strumentale per l'aumentazione interpretativa; 4) fornire strumenti pedagogici per la formazione dell'interprete contemporaneo; 5) dimostrare l'applicabilità del metodo attraverso l'analisi fenomenologica di \textit{Luz} di Guaccero.

% 4. Metodologia
\textbf{Metodologia:} La metodologia integra tre dimensioni co-costitutive: \textit{analisi fenomenologica} di \textit{Luz} mediante mappatura delle 24 tipologie timbriche e documentazione del ``silenzio udibile animato''; \textit{hacking strumentale} con protocolli di aumentazione che trasformano la tecnologia in estensione della corporeità interpretativa; \textit{documentazione riflessiva} dei processi cognitivi attraverso sistemi di notazione delle relazioni emergenti gesto-suono. Il metodo si fonda sulla tensione husserliana \textit{noesis}/\textit{noema}, mantenendo l'interpretazione come luogo della conoscenza anziché sua applicazione.

% 5. Risultati
\textbf{Risultati:} La ricerca produce: una \textit{grammatica dell'ascolto analitico} con catalogazione sistematica delle trasformazioni timbriche in \textit{Luz}; protocolli replicabili di hacking strumentale per l'aumentazione interpretativa; un sistema di notazione per documentare l'emergenza di relazioni interpretative inedite; metodologie pedagogiche che superano la separazione teoria/prassi. L'analisi di \textit{Luz} rivela come il ``silenzio udibile animato'' generi nuove possibilità interpretative, dimostrando l'efficacia del metodo proposto.

% 6. Implicazioni
\textbf{Implicazioni:} Il contributo trasforma i paradigmi didattici dell'interpretazione musicale, fornendo strumenti metodologici per integrare ricerca contemporanea e formazione istituzionale. L'\textit{analisi interpretativa} apre nuove prospettive per la pedagogia musicale, superando l'empirismo della tradizione orale. La sistematizzazione proposta è trasferibile ad altro repertorio contemporaneo, contribuendo a un'archeologia del presente musicale dove l'interprete media tra tradizione e innovazione.

\vspace{1cm}

% PAROLE CHIAVE
\textbf{Parole chiave:} analisi interpretativa, fenomenologia dell'interpretazione, organologia aumentata, spettromorfologia, hacking strumentale

\vspace{1cm}

\clearpage

% VERSIONE CONDENSATA 300 PAROLE
\section*{Abstract - Versione Condensata (300 parole)}

\textbf{Problema e Obiettivi:} La ricerca affronta la separazione tra analisi musicale ed esecuzione interpretativa, proponendo l'\textit{analisi interpretativa} come metodologia unificata. Obiettivo: sistematizzare una coscienza interpretativa che emerga dall'atto stesso dell'interpretazione, sviluppando strumenti per la formazione dell'interprete contemporaneo attraverso l'analisi fenomenologica di \textit{Luz} di Domenico Guaccero.

\textbf{Metodologia e Background:} Integrando spettromorfologia \cite{smalley1997spectromorphology}, fenomenologia dell'interpretazione \cite{rognoni1966fenomenologia, ihde2007listening} e organologia aumentata \cite{lupone2003feedback}, la metodologia combina: analisi fenomenologica delle 24 tipologie timbriche di \textit{Luz} e del ``silenzio udibile animato''; hacking strumentale con protocolli di aumentazione; documentazione riflessiva dei processi cognitivi. Il metodo si fonda sulla tensione husserliana \textit{noesis}/\textit{noema}, mantenendo l'interpretazione come luogo della conoscenza.

\textbf{Risultati e Implicazioni:} La ricerca produce una ``grammatica dell'ascolto analitico'' con catalogazione sistematica delle trasformazioni timbriche, protocolli replicabili di aumentazione interpretativa e metodologie pedagogiche che superano la dicotomia teoria/prassi. Il contributo trasforma i paradigmi didattici dell'interpretazione musicale, fornendo strumenti metodologici trasferibili per integrare ricerca contemporanea e formazione istituzionale.

\vspace{1cm}

\clearpage

% BIBLIOGRAFIA ESSENZIALE (massimo 5 titoli)
\section*{Bibliografia}

\begin{enumerate}[label={[\arabic*]}]
    \item Smalley, Denis. 1997. ``Spectromorphology: explaining sound-shapes''. \textit{Organised Sound}, vol. 2, n. 2, pp. 107-126.
    \item Rognoni, Luigi. 1966. \textit{Fenomenologia della musica radicale}. Garzanti.
    \item Ihde, Don. 2007. \textit{Listening and Voice: Phenomenologies of Sound}. SUNY Press.
    \item Lupone, Michelangelo. 2003. ``Feedback and musical interaction''. \textit{Computer Music Journal}, vol. 27, n. 1, pp. 34-48.
    \item Guaccero, Domenico. 1973. \textit{Luz (da ``Descrizione del corpo'')}. Ricordi.
\end{enumerate}

\newpage

% SUGGERIMENTI E PROMEMORIA (mantenuti dal template originale)
\section*{Promemoria per la compilazione}

\subsection*{Date importanti:}
\begin{itemize}
    \item \textbf{10 giugno 2025:} Scadenza invio proposte
    \item \textbf{10 luglio 2025:} Comunicazione risultati
    \item \textbf{11-31 luglio 2025:} Registrazione anticipata (quota ridotta)
    \item \textbf{1-25 agosto 2025:} Registrazione quota intera
\end{itemize}

\subsection*{Specifiche tecniche:}
\begin{itemize}
    \item \textbf{Abstract:} massimo 300 parole
    \item \textbf{Proposta completa:} massimo 500 parole + 2 pagine esempi/figure
    \item \textbf{Bibliografia:} massimo 5 titoli
    \item \textbf{Parole chiave:} esattamente 5
    \item \textbf{Lingue ufficiali:} italiano e inglese
\end{itemize}

\subsection*{Tipologie di presentazione:}
\begin{enumerate}
    \item \textbf{Relazione individuale:} 20 min + 10 min discussione
    \item \textbf{Sessione preorganizzata:} 3-4 relazioni da 20 min ciascuna
    \item \textbf{Relazione-concerto:} 30 min totali + 10 min discussione
\end{enumerate}

\subsection*{Criteri di valutazione (scala 1-5):}
\begin{itemize}
    \item Coerenza con il tema scelto
    \item Consistenza e pertinenza del background teorico
    \item Chiarezza degli obiettivi
    \item Rigore metodologico
    \item Originalità e interesse dei risultati
    \item Potenziale impatto nella comunità scientifica
    \item Chiarezza del linguaggio e stile di scrittura
\end{itemize}

\subsection*{Invio proposta:}
\begin{itemize}
    \item \textbf{Modulo Google:} \texttt{https://forms.gle/rxqUnbBTp4QnxmPP6}
    \item \textbf{Contatti:} \texttt{convegnoannuale@gatm.it}
    \item \textbf{Sito web:} \texttt{https://www.gatm.it}
\end{itemize}

\end{document}
